\documentclass{protokol}

\usepackage[czech]{babel}
\usepackage[utf8]{inputenc}
\usepackage{icomma}

% Plovouci bloky (obrazky, tabulky)
\usepackage{floatrow}
\floatsetup[table]{capposition=top}
\floatsetup[figure]{frameset={\fboxsep16pt}}
\usepackage{subcaption}

% Tabulky
\usepackage{tabu}
\usepackage{booktabs}
\usepackage{csvsimple}
\usepackage{multirow}
\usepackage{multicol}

% Jednotky
\usepackage{siunitx}
\sisetup{
	locale               = DE,
	inter-unit-product   = \ensuremath{{}\cdot{}},
	list-units           = single,
	list-separator       = {; },
	list-final-separator = \text{ a },
	list-pair-separator  = \text{ a },
	range-phrase         = \text{ až },
	range-units          = single,
}
\DeclareSIUnit\thomson{Th}
\DeclareSIUnit\amu{u}
\usepackage{amsmath}

% Obvody
\usepackage{circuitikz}

% Obrazky a grafy
% \usepackage{graphicx}
\graphicspath{
	{plots/}
	{build/plots/}
}
\usepackage{epstopdf}
\epstopdfsetup{outdir=./build/plots/}

\usepackage{mhchem}
\usepackage{chemfig}

\usepackage[hidelinks,pdfusetitle]{hyperref}

\usepackage[style=iso-numeric, autocite=superscript, backend=biber,
	sorting=none, sortlocale=cs_CZ]{biblatex}
\addbibresource{references.bib}

\jmenopraktika={Diagnostické metody I}  % jmeno predmetu
\jmeno={Jan Slaný}                             % jmeno mericiho
\obor={F}                               % zkratka studovaneho oboru
\skupina={Út 10:00}                     % doba vyuky seminarni skupiny
\rocnik={IV}
\semestr={I}

\cisloulohy={02}
\jmenoulohy={Korpuskulární diagnostika}

\datum={29. listopadu 2022}                  % datum mereni ulohy
\tlak={}% [hPa]
\teplota={}% [C]
\vlhkost={}% [%]

\newcommand\mz{m/z}

\begin{document}
\header

\section{Určení neznámé látky}
\label{sec:unknown}
Posledním úkolem je určit neznámou kapalnou látku z~hmotnostního spektra
jejích výparů.
O~kapalině je známo, že obsahuje uhlík, vodík a kyslík.

Hmotnostní spektrum bylo změřeno pro čtyři energie inoizačních elektronů:
\SIlist[list-separator={, }]{70;40;25;15}{\electronvolt}.
Tato spektra s~vyznačenými maximy jsou na obrázku č.~\ref{fig:unknown-all}.
Nejvyšší detekovaná hodnota $\mz$ je rovna \SI{57}{\thomson},
pročež je možno usoudit, že nominální hmotnost neznámé molekuly
je $\SI{57}{\amu}$.
V~databázi NIST~\parencite{nist} byly vyhledány sloučeniny s~touto hmotností.
Při porovnání zde uvedených hmotnostních spekter s~naměřenými vyšlo najevo,
že neznámou látkou je pravděpodobně cyklopropylamin:
\begin{center}
	\chemfig{H_2N-*3(---)}
\end{center}

Na obrázku č.~\ref{fig:cyclopropylamine-nist} je referenční spektrum
z~databáze NIST \parencite{nist}.
Pravděpodobné přiřazení částic k~jednotlivým maximům je
v~tabulce č.~\ref{tab:unknown}.

\begin{figure}[htp]
	\centering
	\input{plots/unknown-all}
	\caption{Hmotnostní spektrum neznámého plynu při různých energiích.}
	\label{fig:unknown-all}
\end{figure}

\begin{figure}[htp]
	\centering
	\input{plots/cyclopropylamine-nist}
	\caption{Hmotnostní spektrum cyklopropylaminu podle \cite{nist}.}
	\label{fig:cyclopropylamine-nist}
\end{figure}

\begin{table}
	\centering
	\caption{Určené částice}
	\label{tab:unknown}
	\begin{tabular}{ccc}
		\toprule
		$\mz\,[\si{\thomson}]$ & částice \\
		\midrule
		18 & \ce{H2O}\,? \\
		26 & \ce{CN} \\
		27 & \ce{CHN} \\
		28 & \ce{C2H4} \\
		29 & \ce{CH3N} \\
		30 & \ce{CH4N} \\
		32 & \ce{O2}? \\
		\bottomrule
	\end{tabular}
	\qquad
	\begin{tabular}{ccc}
		\toprule
		$\mz\,[\si{\thomson}]$ & částice \\
		\midrule
		39 & \ce{C2HN} \\
		41 & \ce{C3H5} \\
		42 & \ce{C3H6} \\
		52 & \ce{C3H2N} \\
		54 & \ce{C3H4N} \\
		56 & \ce{C3H6N} \\
		57 & \ce{C3H7N} \\
		\bottomrule
	\end{tabular}
\end{table}

\end{document}

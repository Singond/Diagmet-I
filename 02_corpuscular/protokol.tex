\documentclass{protokol}

\usepackage[czech]{babel}
\usepackage[utf8]{inputenc}
\usepackage{icomma}

% Plovouci bloky (obrazky, tabulky)
\usepackage{floatrow}
\floatsetup[table]{capposition=top}
\floatsetup[figure]{frameset={\fboxsep16pt}}
\usepackage{subcaption}

% Tabulky
\usepackage{tabu}
\usepackage{booktabs}
\usepackage{csvsimple}
\usepackage{multirow}
\usepackage{multicol}
\usepackage{pgfplotstable}
\pgfplotstableset{
	header = false,
	skip first n = 1,
	every head row/.style = {
		before row = \toprule,
		after row = \midrule
	},
	every last row/.style = {
		after row = \bottomrule
	},
	use comma,
	set thousands separator = {},
}
\newcommand\massspectable[1]{\pgfplotstableset{
	create on use/species/.style = {
		create col/set list={#1}
	},
	columns={0, species, 2},
	columns/0/.style = {
		column name = $\mz\,[\si{\thomson}]$
	},
	columns/2/.style = {
		column name = $\intensrel$,
		column type = {S[
			table-format = 3.2,
			round-mode = figures,
			round-precision = 2,
			exponent-mode = fixed,
			fixed-exponent = 0,
		]},
		multiply with = 100,
		string type,
	},
	columns/species/.style = {
		column name = částice,
		string type
	},
}}

% Jednotky
\usepackage{siunitx}
\sisetup{
	locale               = DE,
	inter-unit-product   = \ensuremath{{}\cdot{}},
	list-units           = single,
	list-separator       = {; },
	list-final-separator = \text{ a },
	list-pair-separator  = \text{ a },
	range-phrase         = \text{--},
	range-units          = single,
}
\DeclareSIUnit\thomson{Th}
\DeclareSIUnit\amu{u}
\DeclareSIUnit\sccm{sccm}
\usepackage{amsmath}

% Obvody
\usepackage{circuitikz}

% Obrazky a grafy
% \usepackage{graphicx}
\graphicspath{
	{plots/}
	{build/plots/}
}
\usepackage{epstopdf}
\epstopdfsetup{outdir=./build/plots/}

\usepackage{mhchem}
\usepackage{chemfig}

\usepackage[hidelinks,pdfusetitle]{hyperref}

\usepackage[style=iso-numeric, autocite=superscript, backend=biber,
	sorting=none, sortlocale=cs_CZ]{biblatex}
\addbibresource{references.bib}

\jmenopraktika={Diagnostické metody I}  % jmeno predmetu
\jmeno={Jan Slaný}                             % jmeno mericiho
\obor={F}                               % zkratka studovaneho oboru
\skupina={Út 10:00}                     % doba vyuky seminarni skupiny
\rocnik={IV}
\semestr={I}

\cisloulohy={02}
\jmenoulohy={Korpuskulární diagnostika}

\datum={29. listopadu 2022}                  % datum mereni ulohy
\tlak={}% [hPa]
\teplota={}% [C]
\vlhkost={}% [%]

\DeclareSIUnit\arbu{a.~u.}

\newcommand\mz{m/z}
\newcommand\en{E}
\newcommand\intens{I}
\newcommand\intensrel{\intens_\mathrm{r}}

\title{Korpuskulární diagnostika}
\author{Jan Slaný}

\begin{document}
\header

\section{Úvod}
\label{sec:intro}
Hmotnostní spektrometrie je důležitá metoda pro analýzu chemického složení.
Jejím základem je měření poměru hmotnosti a~elektrického náboje iontů
získaných ze vzorku, čehož lze využít k~určení molekulové hmotnosti,
sumárního vzorce či struktury molekuly,
případně zastoupení jednotlivých izotopů.

Přístroj umožňující taková měření se nazývá hmotnostní spektrometr.
Obvykle je tvořen zdrojem iontů, hmotnostním filtrem (analyzátorem)
a~detektorem.

Všechny metody hmotnostní spektrometrie dokážou měřit pouze nabité částice,
proto na začátku měření stojí tvorba iontového svazku ze vzorku.
Existují různé způsoby, jak ze vzorku ionty získat, například
elektrickou ionizací, chemickou ionizací, primárním svazkem iontů či atomů,
laserovou desorpcí nebo elektrosprejem.
V~některých vzorcích jsou ionty již přirozeně přítomny a externí ionizace
tedy není potřeba, to je případ plazmatu.

Ústřední částí spektrometru je hmotnostní filtr, tedy zařízení,
které z~iontového svazku osamostatní částice s~určitou hmotností.
Může tak činit na základě kinetické energie, jako je tomu v~případě
elektrického sektoru; podle hybnosti jako u~magnetického sektoru,
nebo podle poměru hmotnosti a~elektrického náboje $\mz$.
Do poslední kategorie spadají kvadrupólové analyzátory a~iontové pasti.

Koncovou součástí je detektor, který snímá počet částic prošlých hmotnostním
filtrem.
Běžnými detektory jsou například Faradayův pohár,
který měří velikost náboje uloženého dopadajícími ionty,
elektronový násobič nebo elektrooptický iontový detektor,
které zesilují signál tvorbou sekundárních elektronů,
či proudový detektor.

\section{Aparatura}
\label{sec:setup}
Měření bylo prováděno na spektrometru Hiden EQP 500 připojenému k~plazmovému
reaktoru, který využívá kapacitně vázaný radiofrekvenční výboj pro chemickou
depozici z~plynné fáze.
Uvnitř reaktoru jsou dvě rovnoběžné elektrody, horní zemněná a~dolní buzená.
Tubus spektrometru směřuje do prostoru mezi těmito dvěma elektrodami.
Zdroj radiofrekvenčního napětí je generátor Cito~136 (COMET) o~frekvenci
\SI{13.56}{\mega\hertz} a~maximálním příkonu \SI{600}{\watt}.
Proud je k~elektrodě veden přes přizpůsobovací člen, který automaticky
ladí svou impedanci tak, aby minimalizoval výkon odražený zpět do generátoru.
K~měření dodávaného výkonu slouží sonda Octiv VI (Impedans).

Komora je čerpána turbomolekulární vývěvou Pfeiffer HiPace 300
předčerpávanou rotační olejovou vývěvou, mezní tlak je \SI{5e-5}{\pascal}.
Tlak v~reaktoru snímá kapacitron MKS Baratron a~kompaktní manometr
Pfeiffer PKR~251.
Plyny proudí do reaktoru otvory v~horní elektrodě,
jejich množství je regulováno průtokoměry MKS.

\section{Experimenty}
\label{sec:experimental}

\subsection{Hmotnostní spektrum zbytkové atmosféry}
\label{sec:residual}
Na úvod experimentální části bylo změřeno hmotnostní spektrum
zbytkové atmosféry v~aparatuře.
Energie ionizačních elektronů byla nastavena na \SI{70}{\electronvolt}.
Jelikož se reaktor bežně používá k~depozici blíže neurčené sloučeniny,
lze ve~spektru očekávat kromě složek vzduchu také její malé množství.
Povaha této látky bude určena v~kapitole \ref{sec:unknown}.

Naměřené spektrum je na obrázku č.~\ref{fig:residual},
určení jednotlivých částic je v~tabulce č.~\ref{tab:residual}.
Je dobře patrna přítomnost vodní páry, dusíku a kyslíku.
V~menším množství se vyskytuje též argon, oxid uhličitý
a~neznámá částice s~nominální hmotností $\mz = \SI{56}{\thomson}$.
Dusík je možno pozorovat také v~atomární podobě
na~hodnotách $\mz = \SIlist{14;15}{\thomson}$,
které odpovídají izotopům \ce{^{14}N} a~\ce{^{15}N}.
Relativní zastoupení izotopu \ce{^{14}N} na~Zemi je asi~\SI{99.6}\percent,
obě čáry však mají srovnatelnou intenzitu.
Lze proto předpokládat, že čára $\mz = \SI{15}\thomson$ je tvořena
především ionty \ce{NH+}.

\begin{figure}[p]
	\centering
	\input{plots/residual}
	\caption{Hmotnostní spektrum zbytkové atmosféry.}
	\label{fig:residual}
\end{figure}

\begin{table}[hb]
	\centering
	\caption{Hlavní složky zbytkové atmosféry.
		Relativní intenzita $\intensrel$ je vztažena k~maximální
		naměřené intenzitě.}
	\label{tab:residual}
	\massspectable{
		\ce{N+},
		{\ce{NH+}, \ce{^{15}N+}},
		\ce{O+},
		\ce{OH+},
		\ce{H2O+},
		\ce{N2+},
		\ce{O2+},
		\ce{Ar+},
		\ce{CO2+},
		\ce{C3H6N}
	}
	\pgfplotstabletypeset[
		select equal part entry of={0}{2}
	]{results/residual.csv}
	\qquad
	\pgfplotstabletypeset[
		select equal part entry of={1}{2}
	]{results/residual.csv}
\end{table}

\subsection{Hmotnostní spektrum argonu}
\label{sec:argon}
Dále bylo určeno hmotnostní spektrum argonu.
Do reaktoru byl pouštěn argon v~množství \SI{10}{\sccm},
tlak byl přitom nastaven na \SI{6}{\pascal}.
Energie ionizačních elektronů činila stále \SI{70}{\electronvolt}.
Tlak v~analyzátoru byl udržován na \SI{2.4e-4}{\pascal}.

Výsledky měření jsou na obrázku č.~\ref{fig:argon}
a v~tabulce č.~\ref{tab:argon}.
Podle očekávání má nejvýraznější maximum argon o~$\mz = \SI{40}{\thomson}$,
kromě něj však lze pozorovat další částice.
Na hodnotě $\mz = \SI{36}{\thomson}$ je pravděpodobně zachycen \ce{^{36}Ar+}\!.
Relativní zastoupení tohoto izotopu na~Zemi je přibližně \SI{0.3}\percent,
což dobře odpovídá naměřené relativní intenzitě čáry
$\intensrel = \SI{0.42}\percent$.

\begin{figure}[p]
	\centering
	\input{plots/argon}
	\caption{Hmotnostní spektrum argonu.}
	\label{fig:argon}
\end{figure}

\begin{table}[hb]
	\centering
	\caption{Složky naměřeného spektra argonu.}
	\label{tab:argon}
	\massspectable{
		\ce{^{40}Ar^2+},
		\ce{N2+},
		\ce{O2+},
		\ce{^{36}Ar+},
		\ce{^{40}Ar+}
	}
	\pgfplotstabletypeset{results/argon.csv}
\end{table}

\subsection{Ionizační práh argonu}
\label{ionization}
V~této části byla pozorována závislost počtu částic s~určitým $\mz$
zachycených hmotnostním filtrem na energii ionizačních elektronů.
Hmotnostní analyzátor byl tedy nastaven na pevnou hodnotu $\mz$
a~energie ionizačních elektronů byla měněna v~potřebném rozsahu
(\SIrange{5}{20}{\electronvolt} pro \ce{Ar+}
a~\SIrange{5}{60}{\electronvolt} pro \ce{Ar^2+}).
Podmínky v~komoře byly stejné jako v~předchozím případě,
tedy argon o~tlaku \SI{6}{\pascal} při průtoku \SI{10}{\sccm}.

Pokus byl opakován pro dvě hodnoty $\mz$:
pro \SI{40}{\thomson}, což odpovídá iontu \ce{Ar+},
a~pro \SI{20}{\thomson}, tedy iont \ce{Ar^2+}.
Náběhová část změřených průběhů byla poté aproximována polynomem
druhého stupně a~jeho průsečík s~osou $x$ byl prohlášen za~ionizační
energii.

Výsledky jsou na obrázku č.~\ref{fig:ionization}.
Ionizační energie \ce{Ar+} se pohybuje kolem \SI{16.26}{\electronvolt},
u~iontu \ce{Ar^2+} je podle očekávání vyšší, asi \SI{45.65}{\electronvolt}.
Tabulkové hodnoty jsou \SI{15.76}{\electronvolt},
respektive \SI{43.39}{\electronvolt}.
Je také nutno poukázat na skutečnost, že intenzita signálu pro \ce{Ar^2+}
je o~řád nižší než pro jedenkrát ionizovaný argon,
což svědčí o~nižší koncentraci vyšších iontů.

\begin{figure}[htp]
	\centering
	\input{plots/ionization-1}%
	\input{plots/ionization-2}
	\caption{Určení ionizačního prahu argonových iontů \ce{Ar+} a~\ce{Ar^2+}.}
	\label{fig:ionization}
\end{figure}

\subsection{Hmotnostní spektrum plazmatu}
\label{plasma}
Následně bylo zkoumáno složení plazmatu v~kapacitně vázaném výboji.
Tlak i~průtok argonu byly stejné jako v~předchozích případech,
tedy \SI{6}{\pascal} a~\SI{10}{\sccm}.
Výboj byl buzen střídavým napětím o~frekvenci \SI{13.56}{\mega\hertz},
dodávaný výkon činil \SI{20}{\watt}.
Skrz okénko reaktoru bylo možno pozorovat, že výboj hoří a~plazma svítí.
Ionizační komora byla při tomto měření vypnuta,
byly tedy zaznamenávány pouze ionty z~plazmatu.

Naměřené hmotnostní spektrum je na obrázku č.~\ref{fig:plasma}
a~určené složky v~tabulce č.~\ref{tab:plasma}.

\begin{figure}[htp]
	\centering
	\input{plots/plasma}
	\caption{Hmotnostní spektrum iontů z~plazmatu.}
	\label{fig:plasma}
\end{figure}

\begin{table}
	\centering
	\caption{Určené složky hmotnostního spektra plazmatu.}
	\label{tab:plasma}
	\massspectable{
		\ce{H+},
		\ce{OH+},
		\ce{H2O+},
		\ce{H3O+},
		\ce{N2+},
		\ce{CH3N+},
		\ce{O2+},
		\ce{^{36}Ar+},
		\ce{^{38}Ar+},
		\ce{Ar+},
		\ce{ArH+},
		\ce{CO2+},
		\ce{CH3NO+},
		\ce{^{40}Ar2+},
	}
	\pgfplotstabletypeset[
		select equal part entry of={0}{2}
	]{results/plasma.csv}
	\qquad
	\pgfplotstabletypeset[
		select equal part entry of={1}{2}
	]{results/plasma.csv}
\end{table}

\subsection{Energiové spektrum}
\label{sec:energy}
Pro nejvýraznější ionty určené předchozím měřením,
tedy \ce{Ar+}, \ce{ArH+} a~\ce{N2+} byla naměřena energiová spektra.
Hmotnostní filtr byl nastavován na~pevnou hodnotu $\mz$ zkoumaného iontu
a~energiový analyzátor proměřoval energie v~rozsahu
\SIrange{0}{30}{\electronvolt}.
Podmínky v~reaktoru byly stejné jako v~předchozím případě,
tlak tedy činil \SI{6}{\pascal}, ionizační komora zůstala vypnuta.
Měření bylo poté opakováno pro tlak \SI{2.5}{\pascal}.

Výsledky jsou na~obrázku č.~\ref{fig:energy}.
U~jedenkrát ionizovaného argonu je při vyšším tlaku pozorován kromě
celkového snížení intenzity výrazný posun rozdělovací funkce energií doleva.
Dá se to vysvětlit zvýšením srážkové frekvence, které vede k~častější
výměně energie mezi částicemi.
Snižuje se tak pravděpodobnost, že konkrétní částice nabere větší
množství energie, protože ji dříve ve~srážkách předá jiným.

Iont \ce{ArH+} vzniká ve výbojích protonací argonu, tedy připojením
protonu k~argonovému atomu:
\begin{equation}
	\label{eq:arh}
	\ce{Ar} + \ce{H2+} \rightarrow \ce{ArH+} + \ce{H}.
\end{equation}
Za vyššího tlaku u~něj pozorujeme silný pokles intenzity.
Tento jev lze přičíst zvýšené koncentraci neutrálního argonu.
Atom \ce{Ar} a~iont \ce{ArH+} mají podobnou hmotnost a~při pružné srážce
dochází ke~snadné výměně kinetické energie.
Ionty proto při srážkách s~neutrálním argonem ztrácejí energii,
což se projeví pozorovaným poklesem rozdělení.

Rozdělovací funkce \ce{N2+} vykazuje nejmenší změnu,
i~zde je však vidět pokles intenzity a~posun energií k~nižším hodnotám.

Z~rozdělovací energie iontů lze také odhadnout napětí, kterým byly ionty
urychleny, tedy velikost plazmového potenciálu.
Nejvyšší zaznamenaná energie pro tlak \SI{2.5}{\pascal} je zhruba
\SI{27}{\electronvolt}, plazmový potenciál je tedy kolem \SI{27}{\volt}.
Při tlaku \SI{6}{\pascal} klesá tato hodnota asi na~\SI{25}{\volt}.

\begin{figure}[htp]
	\centering
	\input{plots/plasma-ar}%
	\input{plots/plasma-arh}
	\input{plots/plasma-nn}
	\caption{Energiová spektra nejvíce zastoupených iontů v~plazmatu.}
	\label{fig:energy}
\end{figure}

\subsection{Určení neznámé látky}
\label{sec:unknown}
Posledním úkolem je určit neznámou kapalnou látku z~hmotnostního spektra
jejích výparů.
O~kapalině je známo, že obsahuje uhlík, vodík a kyslík.

Hmotnostní spektrum bylo změřeno pro čtyři energie inoizačních elektronů:
\SIlist[list-separator={, }]{70;40;25;15}{\electronvolt}.
Tato spektra s~vyznačenými maximy jsou na obrázku č.~\ref{fig:unknown-all}.
Nejvyšší detekovaná hodnota $\mz$ je rovna \SI{57}{\thomson},
pročež je možno usoudit, že nominální hmotnost neznámé molekuly
je $\SI{57}{\amu}$.
V~databázi NIST~\parencite{nist} byly vyhledány sloučeniny s~touto hmotností.
Při porovnání zde uvedených hmotnostních spekter s~naměřenými vyšlo najevo,
že neznámou látkou je pravděpodobně cyklopropylamin:
\begin{center}
	\chemfig{H_2N-*3(---)}
\end{center}

Na obrázku č.~\ref{fig:cyclopropylamine-nist} je referenční spektrum
z~databáze NIST \parencite{nist}.
Pravděpodobné přiřazení částic k~jednotlivým maximům je
v~tabulce č.~\ref{tab:unknown}.

\begin{figure}[htp]
	\centering
	\input{plots/unknown-all}
	\input{plots/cyclopropylamine-nist}
	\caption{Hmotnostní spektrum neznámého plynu při různých energiích.
		Hmotnostní spektrum cyklopropylaminu podle \cite{nist}.}
	\label{fig:unknown-all}
	\label{fig:cyclopropylamine-nist}
\end{figure}

\begin{table}
	\centering
	\caption{Částice určené ve~spektru neznámé látky.}
	\label{tab:unknown}
	\massspectable{
		\ce{H2O+},
		{\ce{C2H2+}, \ce{CN+}},
		{\ce{C2H3+}, \ce{CHN+}},
		{\ce{C2H4+}, \ce{CH2N+}},
		\ce{CH3N+},
		\ce{CH4N+},
		\ce{O2+},
		\ce{C2HN+},
		{\ce{C3H5+}, \ce{C2H3N+}},
		{\ce{C3H6+}, \ce{C2H4N+}},
		\ce{C3H2N+},
		\ce{C3H4N+},
		\ce{C3H6N+},
		\ce{C3H7N+},
	}
	\pgfplotstabletypeset[
		select equal part entry of={0}{2}
	]{results/unknown.csv}
	\qquad
	\pgfplotstabletypeset[
		select equal part entry of={1}{2}
	]{results/unknown.csv}
\end{table}

\section{Závěr}
Pomocí metod hmotnostní spektroskopie bylo zkoumáno několik jevů.
Bylo změřeno hmotnostní spektrum zbytkové atmosféry, které odhalilo
především složky vzduchu a~malé množství látky používané v~aparatuře
k~depozici.
Dále bylo změřeno hmotnostní spektrum a~ionizační energie argonu.
Spektrum argonu odpovídalo očekávání, zjištěné ionizační energie
byly odchýleny o~méně než 5\,\% od tabulkových hodnot.
Bylo zkoumáno hmotnostní spektrum kapacitně vázaného výboje v~argonu
a~energiové spektrum jeho nejvýraznějších složek.
Z~nich byl také odhadnut plazmový potenciál.
Nakonec byla analyzována neznámá organická látka.
Podle jejího hmotnostního spektra bylo usouzeno, že se jedná
o~cyklopropylamin.

\printbibliography[title={Seznam použité literatury}]
\end{document}
